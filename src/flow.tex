The overall strategy is to do the expensive full ME calculations as
infrequently as possible, ideally once for DNN training and once more
for a final pass before publication, with the DNNs utilized as a good
approximation in between. A future analysis flow using the ME method
with DNNs might look something like the following: One performs a
large number of ME calculations using a traditional numerical
integration technique like {\sf VEGAS} or {\sf FOAM} on a large CPU
resource like an HPC, Cloud or the Grid, ideally exploiting
acceleration on many-core devices like GPUs or even FPGAs. The DNN
training data is generated from the phase space sampling in performing
the full integration in this initial pass, and DNNs are trained
either \emph{in situ} or \emph{a posteriori}. The accuracy of the
DDN-based ME calculation can be assessed through this procedure. As
the analysis develops and progresses through selection and/or sample
changes, systematic treatment, etc., the DNN-based ME calculations are
used in place of the time-consuming, full ME calculations to make the
analysis nimble and to preserve the ME calculations. Before a result
using the ME method is published, a final pass using full ME
calculation would likely be performed both to maximize the numerical
precision or sensitivity of the results and to validate the analysis
evolution via the DNN-based approximations.


