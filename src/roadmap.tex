There are several activities which are proposed to further develop the
idea of a Sustainable Matrix Element Method. The first is to establish
a cross-experiment group interested in developing the ideas presented
in this Section, along with a common software project for ME
calculations, for example in the spirit of~\cite{MoMEMta}. Given the
nature of the challenges for a sustainable ME method, this is area
which is very well-suited for impactful collaboration with computer
scientists and those working in machine learning, so effort should be
placed in establishing those connections. Using a few test cases
(e.g. $t\bar{t}$ or $t\bar{t}h$ production), evaluation of DDN choices
and configurations, developing methods for DNN training from full ME
calculations and direct comparisons of the integration accuracy
between Monte Carlo and DNN-based calculations should be
undertaken. More effort should also be placed in developing compelling
applications of the ME method for HL-LHC physics. In the longer term
and after successfully demonstrating the value of the methods on a few
test cases, we propose exploring the possibilty of
Sustainable-Matrix-Element-Method-as-a-Service (SMEMaaS) where shared
software and infrastructure could be used through a common API.

