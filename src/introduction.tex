The Matrix Element (ME)
Method~\cite{Kondo:1988yd,Fiedler:2010sg,2011arXiv1101.2259V,Elahi:2017ppe}
is a powerful technique which can be utilzed for measurements of
physical model parameters and direct searches for new phenomena. It
has been used extensively by collider experiments at the Tevatron for
SM measurements and Higgs boson
searches~\cite{Abazov:2004cs,Abulencia:2006ry,Aaltonen:2008mv,Aaltonen:2010cm,Abazov:2009ii,Aaltonen:2009jj}
and at the LHC for measurements in the Higgs and top quark sectors of
the
SM~\cite{Chatrchyan:2012xdj,Chatrchyan:2013mxa,Aad:2014eva,Khachatryan:2015tzo,
Khachatryan:2015ila,Aad:2015gra,Aad:2015upn}. The ME method is based
on \emph{ab initio} calculation of the probabilty density function
$\mathcal{P}$ of an event with observed final-state particle momenta
${\bf x}$ to be due to a physics process $\xi$ with theory parameters
$\boldsymbol\alpha$. One can compute $\mathcal{P}_{\xi}({\bf
x}|{\boldsymbol\alpha})$ by means of the factorization theorem from
the corresponding partonic cross-sections of the hard scattering
process involving parton momenta ${\bf y}$ and is given by 
\begin{equation}
\mathcal{P}_{\xi}({\bf x}|{\boldsymbol\alpha}) = \frac{1}{\sigma^{\rm fiducial}_{\xi}(\boldsymbol\alpha)} \int d\Phi ({\bf y}_{\rm final}) \; dx_1 \; dx_2~\frac{f(x_1)f(x_2)}{2s x_1 x_2} \; |\mathcal{M}_{\xi}({\bf y}|\boldsymbol\alpha)|^2 \; \delta^{4}({\bf y}_{\rm initial}-{\bf y}_{\rm final}) \; W({\bf x}, {\bf y})
\label{eqn:MEProb}
\end{equation}
where and $x_i$ and ${\bf y}_{{\rm initial}}$ are related by $y_{{\rm
initial},i}\equiv \frac{\sqrt{s}}{2}(x_i,0,0,\pm x_i)$, $f(x_i)$ are
the parton distribution functions, $\sqrt{s}$ is the collider
center-of-mass energy, $\sigma^{\rm
fiducial}_{\xi}(\boldsymbol\alpha)$ is the total cross section for the
process $\xi$ (with $\boldsymbol\alpha$) times the detector
acceptance, $d\Phi({\bf y})$ is the phase space density factor,
$\mathcal{M}_{\xi}({\bf y}|\boldsymbol\alpha)$ is the matrix element
(typically at leading-order (LO)), and $W({\bf x}, {\bf y})$ is the
probability density (aka "transfer function") that a selected event
${\bf y}$ ends up as a measured event ${\bf x}$.

One can use calculations of Eqn.~\ref{eqn:MEProb} in a number of ways
to search for new phenomena at particle colliders. For measurement of
model parameters $\boldsymbol\alpha$, one would maximize the
likelihood function for observed events
$\mathcal{L}(\boldsymbol\alpha)$ given by
\begin{equation}
\mathcal{L}(\boldsymbol\alpha) = \prod_{i} \sum_{k} f_k \mathcal{P}_{\xi_k}({\bf x}_i|{\boldsymbol\alpha})
\label{eqn:LH}
\end{equation}
where $f_k$ are the fractions of (non-interfering) processes
contributing to the data. For new particle searchres, one can (using
Bayes' Theorem~\cite{Bayes01011763}) compute for a hypthosized signal
$S$ the probability $P(S|{\bf x})$ given by
\begin{equation}
P(S|{\bf x}) = \frac{\sum_{i} \beta_{S_i} \mathcal{P}_{S_i}({\bf x}|\boldsymbol\alpha_{S_i}) }{\sum_{i} \beta_{S_i} \mathcal{P}({\bf x}|\boldsymbol\alpha_{S_i}) + \sum_{j} \beta_{B_j} \mathcal{P}({\bf x}|\boldsymbol\alpha_{B_j})}
\label{eqn:LR}
\end{equation}
where, $S_i$ and $B_j$, denote all signal and background processes
relevant to the considered phase space and $\beta$ are the \emph{a
priori} expected process fractions. According to the Neyman-Pearson
Lemma~\cite{Neyman289}, Eqn.~\ref{eqn:LR} is the optimal discriminant
function for $S$ in the presence of $B$ and can be used to extract a
signal fraction in the data.


