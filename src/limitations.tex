One drawback to the ME Method is that it has traditionally relied on
LO matrix elements, although nothing in principle limits the ME method
to LO calculations. Techniques that accomodate initial-state QCD
radiation within the LO ME framework using transverse boosting and
dedicated transfer functions to integrate over the transverse momentum
of initial-state partons have been developed~\cite{Alwall:2010cq}.
Another challenge is development of the transfer functions which rely
on tediously hand-crafted fits to full simulated Monte-Carlo events.

The most serious difficulty in the ME method, and the one which has
limited its applicability to searches for beyond-the-SM physics and
precision measurements at collider experiments, is that it is
very \emph{computationally intensive}. If this limitation could be
overcome, then it would enable more widespread use of ME methods for
analysis of LHC data. This could be particularly important for
extending the new physics reach of the HL-LHC which will be dominated
by increases in integrated luminosity rather than center-of-mass
collision energy.

Accurate evaluation of Eqn.~\ref{eqn:MEProb} is computationally
challenging primarily for two reasons: (1) it involves
high-dimensional integration over a large number of events, signal and
background hypotheses, and systematic variations and (2) it involves
sharply-peaked integrands\footnote{a consequence of imposing
energy/momentum conservation in the processes} over a large domain in
phase space. In reference to point (1), the matrix element
$\mathcal{M}_{\xi}({\bf y}|\boldsymbol\alpha)$ in the method involves
all partons in the $n\rightarrow m$ process, so when the 4-momentum of
particles are not completely measured experimentally (e.g. neutrinos),
one must integrate over the missing information which increases the
dimensionality of the integration. In reference to point (2), a clever
technique to re-map the phase space in order to reduce the sharpness
of integrate in that space in an automated way ({\sf
MADWEIGHT}~\cite{Artoisenet:2010cn}) is often used in conjunction with
a matrix element calculation package ({\sf
MADGRAPH\_aMC\@NLO}~\cite{Alwall:2014hca}). In practice, evaluation of
definite integrals by the ME approach invokes techniques such as
importance sampling (see {\sf
VEGAS}~\cite{PETERLEPAGE1978192,Ohl:1998jn} and {\sf
FOAM}~\cite{JADACH200355}) or recursive stratified sampling (see
MISER~\cite{Press:1989vk}) Monte Carlo integration. Acceleration of
some of these techniques on modern computing architectures has been
achieved, for example concurrent phase space sampling in VEGAS on
GPUs.


